% !TEX TS-program = xelatex
% !TEX encoding = UTF-8 Unicode
% File acl-ijcnlp2009.tex
%
% Contact  jshin@csie.ncnu.edu.tw
%%
%% Based on the style files for EACL-2009 and IJCNLP-2008...
%% Based on the style files for EACL 2006 by 
%%e.agirre@ehu.es or Sergi.Balari@uab.es
%% and that of ACL 08 by Joakim Nivre and Noah Smith

\documentclass[11pt]{article}
\usepackage{acl-ijcnlp2009}
\usepackage[	
   pdfdisplaydoctitle, breaklinks, colorlinks, linkcolor=black, citecolor=black, filecolor=black, urlcolor=black, 
   backref, hyperfootnotes]{hyperref} % backref a modre URL asi nakonec zrusime 
%\usepackage{times}
\usepackage{url}
\usepackage{amsmath}
\usepackage{color} %pro korektury
\usepackage{paralist} % for better itemize and enumerate

% a footer required for the first page
\usepackage{fancyhdr}
\fancyhead{} % clear all header fields
\renewcommand{\headrulewidth}{0pt}
\fancyfoot[C]{Proceedings of ICON-2009: 7th International Conference on Natural Language Processing, Macmillan Publishers, India. Also accessible from http://ltrc.iiit.ac.in/proceedings/ICON-2009}

% xelatex
\usepackage{fontspec, xunicode, xltxtra}
\defaultfontfeatures{Mapping=tex-text}
\setmainfont{Times New Roman}
\setmonofont[Scale=MatchLowercase]{Luxi Mono}
\setmathsf{Lohit Hindi}%\XXX
\newfontinstance\hi[Script=Devanagari]{Lohit Hindi}
\newcommand{\hindi}[1]{{\hi #1}}
\newfontinstance\tl{Gentium}
\newcommand{\translit}[1]{{\tl \textit{#1}}}

% natbib
\usepackage{natbib}
\bibliographystyle{plainnat}
\bibpunct{(}{)}{;}{a}{,}{,}

% our defs
\def\perscite#1{\citet{#1}}  
\def\parcite#1{\citep{#1}} 
%ps: Did you mean \citep and \citet (in-parentheses and textual reference)? There is even more, see `texdoc natbib`.
%obo: your def seems right.
\def\Sref#1{Section~\ref{#1}}
\def\Tref#1{Table~\ref{#1}}
\def\Fref#1{Figure~\ref{#1}}
\newcommand{\red}[1]{\textcolor{red}{#1}} % komentare (TODO)
\newcommand{\XXX}{\textcolor{red}{XXX }} % komentare (TODO)


\title{English-Hindi Translation -- Obtaining Mediocre Results\\with Bad Data and
Fancy Models%
\thanks{ \hspace{.6em}The research has been supported by the grants GAAV ČR 1ET201120505 (Grant Agency of the Academy of Sciences of the Czech Republic), 
MSM0021620838 (Czech Ministry of Education),
FP7-ICT-2007-3-231720 (EuroMatrix Plus), and GAUK 4307/2009 (Grant agency of the Charles University).
}
}

% oboXXX nedat Gaurava az za Michaly?
% dz: jsem spíš proti. Konečný přínos Michalů je možná větší, ale pro Gaurava to byla ty 3 měsíce hlavní náplň činnosti a dělal přímo na samotném překladu, ne na "doplňcích".
\author{Ondřej Bojar, Pavel Straňák, Daniel Zeman, Gaurav Jain, Michal Hrušecký, Michal Richter, Jan Hajič\\
Univerzita Karlova v Praze, Ústav formální a aplikované lingvistiky,\\
Malostranské náměstí 25, 118 00, Praha, Czech Republic\\ 
\texttt{\{bojar,stranak,zeman,hajic\}@ufal.mff.cuni.cz}\\
\texttt{gauravjain@cse.iitb.ac.in}, \texttt{michal@hrusecky.net}, \texttt{michalisek@gmail.com}
}
%end author
%\title{Instructions for ACL-IJCNLP 2009 Proceedings}
%
%\author{First Author\\
%  Affiliation / Address line 1\\
%  Affiliation / Address line 2\\
%  {\tt email@domain}  \And
%  Second Author\\
%  Affiliation / Address line 1\\
%  Affiliation / Address line 2\\
%  {\tt  email@domain}}

\date{}

\begin{document}
\maketitle
\thispagestyle{fancy}

\begin{abstract}
We describe our attempt to improve on previous English to Hindi machine
translation results, using two open-source phrase-based MT systems: Moses and
Joshua. We use several approaches to morphological tagging: from automatic word
classes, through stem-suffix segmentation, to a POS tagger.
%We also experiment
%with factored language models.
We evaluate various combinations of training data
sets and other existing English-Hindi resources. To our knowledge, the BLEU
score we obtained is currently the best published result for the IIIT-TIDES
dataset.
\end{abstract}

\section{Introduction}
\label{sec:intro}

Machine translation is a challenging task and more so with significant
differences in word order of the languages in question and with the target
language explicitly marking more details in word forms than the source language
does. Precisely this holds for the English-Hindi pair we study.

We try to
explore the problems on several fronts: \Sref{sec:data} describes our
careful cleanup and a few additions to the training data. \Sref{morf} is devoted
to several additional variants of morphological representation of the target
side.
%and in \Sref{flm} we study the impact of using factored language models
%\parcite{bilmes:kirchhoff:factored:2003} instead of independent standard
%$n$-gram models for ensuring
%output coherence.
\Sref{joshua} evaluates the impact of using a hiearchical
instead of phrase-based model. \Sref{human} concludes our experiments by
providing a preliminary human evaluation of translation quality.

\section{Data}
\label{sec:data}

\def\microsection#1{{\bf #1.}}

\microsection{Tides}
\label{sec:tides}
A dataset originally collected for the DARPA-TIDES surprise-language contest in
2002, later refined at IIIT Hyderabad and provided for the NLP Tools Contest at
ICON 2008 \citep{nlptoolscontest:2008:summary}: 50K sentence pairs for training,
1K development and 1K test data (1~reference translation per sentence).

The corpus is a general domain dataset with news articles forming the greatest
proportion. It is aligned on sentence level, and tokenized to some extent.
We found the tokenization insufficient
%(e.g. \textit{anglo-american} would be 1
%token instead of 3, making the data sparseness problem more severe)
and ran our own tokenizer on top of it.
Cleaning was also necessary due to significant noise in the data, often caused
by misconversion of Latin script.
    
We used Tides as our primary dataset and all reported scores are measured on its
test data. However, note that due to the processing we applied to both training
and test data, our results are not directly comparable to the results of the
2008 NLP Tools Contest.\footnote{For instance, the Joshua BLEU score of a model
trained on Tides only, as we present it later in \Tref{tab:joshua}, is 12.27 on
the cleaned data, and 11.10 on the raw data; see also \Tref{tab:previous}.} We are happy
to share our cleaning tools to make the experiments reproducible.\footnote{\url{https://wiki.ufal.ms.mff.cuni.cz/pub-company:icon2009}; accept the certificate.}

\microsection{Daniel Pipes}
\label{sec:dpipes}
A journalist Daniel Pipes' website:\footnote{\url{http://www.danielpipes.org/}} limited-domain articles about the Middle East. 
The articles are originally written in English, many of them are translated to
up to 25 other languages, including Hindi (322 articles, 6,761 sentence
pairs).

\microsection{Emille}
\label{sec:emille}
EMILLE corpus \citep{baker:2002} consists of three components: monolingual,
parallel and annotated corpora.
%There are fourteen monolingual corpora,
%including both written and (for some languages) spoken data for fourteen South
%Asian languages.''
%: Assamese, Bengali, Gujarati, Hindi, Kannada, Kashmiri,
%Malayalam, Marathi, Oriya, Punjabi, Sinhala, Tamil, Telegu and Urdu.
%The EMILLE
%monolingual corpora contain approximately 92,799,000  words (including 2,627,000
%words of transcribed spoken data for Bengali, Gujarati, Hindi, Punjabi and
%Urdu).''
The parallel corpus consists of 200,000 words of text in English and
its accompanying translations in Hindi and other languages. Whenever we mention
Emille, we mean the parallel English-Hindi section.

The original Emille turned out to be very badly aligned and had spelling errors,
so we worked with a manually cleaned and aligned subset of 3,501 sentence
pairs.\footnote{We are very grateful to Om Dammani and his colleagues from IIT
Mumbai for making their corrected subset of Emille available to us.}

%\subsection{Other Data}
\label{sec:data-o}

We also tried various other small datasets:

 \microsection{ACL 2005}
 A subset of Emille, used in the shared task of the
 ACL 2005 workshop on ``Building and Using Parallel Texts: Data Driven Machine
 Translation and Beyond''.\footnote{Downloaded from the workshop website at
 \url{http://www.cse.unt.edu/~rada/wpt05/}. We used this dataset on the
 assumption that it might be better aligned, compared to the original Emille
 parallel corpus.}

 \microsection{Wiki NEs} 
 We have extracted English titles of Wikipedia
 entries, that are translations (or rather transcriptions) of Hindi, Marathi and
 Sanskrit named entities (names of persons, artifacts, places, etc.), and their
 translations in Devanagari.
% We used XML abstracts of English Wikipedia; the
% begining of an entry looks something like this, so it is easy to parse it with
% a regular expression: 
 
%\hspace{1em} \textbf{Mumbai} (Marathi: \hindi{मुंबई}, \textit{Mumbaī}\ldots 
 
% We took not only Hindi, but also Marathi and Sanskrit entities on the
% assumption that these are also commonly used in Hindi texts. See for example
% Mumbai (Bombay) above.
 
 \microsection{Shabdanjali}
 A GPL-licensed collaborative English-Hindi dictionary, containing about 26,000
 entries, available on the web.
% Available on the web in various formats and encodings. It is formatted
% similarly to printed bilingual dictionaries, so several filters had to be
% applied to refine a simple list of word pairs out of it.

 \microsection{Agriculture domain parallel corpus}
 English-Hindi-Marathi-UNL parallel corpus from Resource Center for Indian
 Language Technology
 Solutions.\footnote{\url{http://www.cfilt.iitb.ac.in/download/corpus/parallel/agriculture_domain_parallel_corpus.zip}}
 It contains 17,105 English and 13,248 Hindi words.
 
The impact of adding additional data to the Tides training data is partially
illustrated by \Tref{tab:joshua} and \Tref{tab:human:more}. Adding other data
has similar results.

\subsection{Normalization}

As always with statistical approaches, we want our training data to match the
test data as closely as possible. There are style variations throughout our data
that can and should be normalized automatically. For instance, the Tides corpus
usually (except for some conversion errors) terminates a sentence with the
period (``.''). However, some of our additional data sets use the traditional
Devanagari sign called \textit{danda} (``\hindi{।}'') instead. Our normalization
procedure replaces all dandas by periods, converts Devanagari digits to ASCII
digits and performs other minor reductions of the character set (e.g.
normalizing
non-ASCII punctuation). Some changes affect the English data as well. 

\subsection{Lessons Learnt}

Hindi as the target language possesses some features that negatively influence
MT performance: richer morphology than English, greater noise in training data
and harder sparse-data problem due to vocabulary that combines words from
various etymological sources.

One common cause of data sparseness is unstable orthography of English and other
loanwords (or even transcribed citations), cf. the following counterparts of the
English word ``standards'', all present in the data:

\begin{compactitem}
\item \hindi{स्टैंडर्डज} \textit{(sṭaiṁḍarḍaja)} \translit{(sṭaiṁḍarḍaja)}
\item \hindi{स्टैंडर्डस} \textit{(sṭaiṁḍarḍasa)}
\item \hindi{स्टैंडर्ड्स} \textit{(sṭaiṁḍarḍsa)}
\end{compactitem}

Also common is the case where genuine English text has been (during some
processing at the site of the data provider) interpreted as Devanagari encoded
using Latin letters. Thus, \hindi{ईन्ङोर्मटिओन् छोम्मिसिओनेर्} \textit{(īnṅormaṭion
chommisioner)} should actually read (even in the Hindi text) \textit{Information
Commis(s)ioner}. If transcribed to Devanagari, it would be
% Zdá se, že font Madan neobsahuje nuktu ani znaky s ní.
% i+n+virám+f+o...
%\hindi{इन्फ़ोर्मेशन कोमिशनेर}
% i+n+virám+ph+nukta+o...
\hindi{इन्फ़ोर्मेशन कोमिशनेर}
\textit{(informeśana komiśanera)}.

\section{Target-Side Morphology}
\label{morf}
Richer morphology on the target side means that besides selecting the
\textit{lexically correct} Hindi equivalent of an English word (such as
\textit{book $\rightarrow$} \hindi{किताब} \textit{(kitāba)}), the system also
must correctly guess the \textit{grammatical features} (such as the direct vs.
oblique case in \textit{books $\rightarrow$} \hindi{किताबें} \textit{(kitābeṁ)} vs.
\hindi{किताबों} \textit{(kitāboṁ)}).
% Tu následující větu o shodě lze při nedostatku místa vynechat.
Moreover, grammatical agreement (such as
\hindi{मॆरी किताब} \textit{(merī kitāba)} ``my book'' vs. \hindi{मॆरा कमरा}
\textit{(merā kamarā)} ``my room'') further multiplies possible translations
of an English word.
A separate model of target-language morphology may help the system select the
correct translation.

We have tried two supervised and four unsupervised approaches to Hindi
morphology, see below.
%
%a full POS tagging
%with a morphology and POS tagger from IIT Mumbai (\Sref{sec:tagger}) and
%suffixes learned from a textbook (\Sref{sec:textbook}). We also tried several
%unsupervised methods. A short comparison of the morphological factors is given
%in the \Tref{tab:morph}.
%
The results in \Tref{tab:morph} and \Tref{tab:human} show that we were not able
to achieve any improvements by using real POS tags
%(see \Sref{sec:tagger})
compared to automatic word classes produced by \texttt{mkcls} or classes created
by hand.
%(\Sref{sec:textbook}).

\subsection{Supervised Morphology} 
\label{sec:pos}

\microsection{POS tagger}
\label{sec:tagger}
The only Hindi POS tagger we were able to find on the web, download and use is a part of the
GATE framework \citep{Cunningham:2002rw}. That is however more of a
demonstration, than a real tagger.
%Other than that, we found no usable tagger.
%That is probably one of the reasons that lead researchers to pursue
%unsupervised
%approaches to morphology; see e.g. \citet{BoStEnglishHindiTranslation2008}.
Fortunately, we also had the opportunity to work with a CRF-based tagger from IIT
Mumbai \citep{Kuhoo-Gupta:2006gd}.\footnote{Many thanks to Pushpak
Bhattacharyya for making this tool available to us.}
%dz 14.10.2009: následující URL možná už nefunguje?
%\footnote{\url{http://www.cfilt.iitb.ac.in/tools/Hindi_POStagger.zip}}

%The java application providing morphological analyses turned out to be a good
%detector of dirty input data, because it crashed on various characters that
%(mostly) did not belong to the input text. Most of these characters probably
%appeared in the input text because of failed transcription as described in
%\Sref{sec:tides}. However, since it crashes on occurrences of \% or \$, or
%square brackets, which morphology uses for its own purpose, it can pose a
%problem when these characters occur legitimately in the input text. 

%Once we cleaned-up and normalized the input data, morphology and POS tagger ran smoothly. 

\microsection{Textbook suffixes}
\label{sec:textbook}
Since the main morphological device of Hindi is alternating word suffixes,
one can model suffixes instead of morphological tags.
Our unsupervised approaches (see below) automatically acquire long but noisy
lists of Hindi suffixes, some of which may bear grammatical meaning.
As an alternative, we compiled a short list of 31 suffixes (including
duplicates) for regular declension of nouns, verb formation etc.
The cost of such a list was negligible: one of the authors (who does not speak
Hindi) spent less then two hours with a Hindi textbook
\citep{teach_yourself_hindi}, looking for grammar-describing sections.

\subsection{Unsupervised Morphology} 
We have tried two simple unsupervised methods:
\begin{itemize}
\item Automatic classes created by \texttt{mkcls}\footnote{\url{http://www.fjoch.com/mkcls.html}} tool contained in GIZA++ installation. 
\item simple n-character long suffixes
\end{itemize}
and two more elaborate methods:
%Affisix and Hindomor.

\label{sec:aff}
{\bf Affisix} is an open source tool for unsupervised recognition of affixes in a
given language. It takes a large list of words and generates a list
of possible suffixes or prefixes scored according to a selected method or a
combination of methods. In this case, we used it to obtain the 100 top scoring
suffixes
using backward entropy and backward difference entropy methods (see
\cite{affisixtsd2008} for details). The longest possible suffix seen in a word
was then treated as the morphological tag of the word, leaving some words with a
blank tag.

\label{sec:hindomor}
{\bf Hindomor} is another tool for unsupervised segmentation of words into
morphemes. It was originally published in the context of information
retrieval \citep{Zeman:2008}.

The tool has been trained on the word types of the Hindi side of the Tides
corpus. For every word the algorithm searches for positions where it can be cut
in two parts: the stem and the suffix. Then it tries to filter the stem and
suffix candidates so that real stems and suffixes remain. The core idea is that
real stems occur with multiple suffixes and real suffixes occur with multiple
stems.

Given the lists of stems and suffixes obtained during training, we want to find
the stem-suffix boundary in a word of the same language. Theoretically, we could
use the learned stem-suffix combinations to require that both stem and suffix be
known. However, this approach proved too restrictive, so we ended up in using
just the list of suffixes. If a word ends in a string equal to a known suffix,
the morpheme boundary is placed at the beginning of that substring.


\begin{table}[t]
\begin{center}
\small
\begin{tabular}{l  l | ll}
factor & BLEU &
factor & BLEU\\
\hline
tag & 12.03±0.75 &	hitbsuf & 11.58±0.74\\
wc50 & 11.97±0.73 &	hindomor2 & 11.55±0.74\\
wc10 & 11.76±0.74 &	hindomor1 & 11.54±0.71\\
lcsuf3 & 11.66±0.75 &	affddf & 11.50±0.7\\
lcsuf1 & 11.63±0.72 &	affbdf & 11.33±0.72\\
hindomor3 & 11.60±0.73 &	lcsuf2 & 11.14±0.74\\
\end{tabular}
\end{center}
\caption{Target side morphology: Using different additional factors for second
language model of MT system and its  effect on BLEU score. Trained on
IIIT-TIDES only. 
tag -- POS tags; wc\textit{n} -- \textit{n} word classes from mkcls; hitbsuf --
word classes created by hand; lcsuf\textit{n} -- simple
n-character suffixes; hindomor\textit{n};
aff\textit{xxx} -- Affisix
} 
\label{tab:morph}
\end{table}


The best BLEU score we achieved with Moses while making experiments with different
factors for target side morphology was 12.22±0.78:
trained on Tides and Daniel Pipes data with \texttt{hitbsuf}
as the additional factor.


\section{Hierarchical Phrase-Based Models}
\label{joshua}

One of the major differences between English and Hindi is the word order.
Massive reordering must take place during translation.
That is why we conducted several experiments with hierarchical
translation models \citep{chiang:2007}, namely with Joshua \citep{joshua:2009},
and compared the results with classical phrase-based models \citep[Moses, ][]{moses:2007}.

Hierarchical phrase-based translation is based on synchronous context-free grammars (SCFG).
Like phrase-based translation, pairs of corresponding source- and target-language phrases
(sequences of tokens) are learnt from training data. The difference is that in hierarchical
models, phrases may contain ``gaps'', represented by nonterminal symbols of the SCFG.
If a source phrase $f$ contains a nonterminal $X_1$, then its translation $e$ also
contains that nonterminal, and the decoder can replace the nonterminal by any phrase $f_1$
and its translation $e_1$, respectively. An illustrative rule for English-to-Hindi translation is

%Zatím neumíme hindštinu uvnitř matematiky, ale neškodilo by, kdybysme ji tam nějak dostali.
%$$X \rightarrow \langle X_1 of X_2 , X_2 \hindi{का} X_1 \rangle$$

\begin{center}
$X \rightarrow \langle X_1\cdot of\cdot X_2\rangle , \langle X_2\cdot \mathsf{का} \cdot X_1 \rangle$
\end{center}

where the Hindi word \hindi{का} \textit{(kā)} is one of several grammatical forms of the Hindi equivalent
of ``of'', and the subscripts on the nonterminals cause the two (noun) phrases around \textit{of}
to be reordered around \hindi{का} in the translation.

Hierarchical models have been known to produce better results than classical phrase-based
models \citep{chiang:etal:2005}.

In our experiments we used the Joshua implementation of the hierarchical phrase-based
algorithms.\footnote{Many thanks to the team at Johns Hopkins University for creating
Joshua and making it publicly available.} We set the maximum phrase length to 5,
MERT worked with 300 best translation hypotheses per iteration, and the number of
iterations was limited to 5.

Unless stated otherwise, our experiments use a trigram language model trained on the
target side of the Tides training data. In accord with \citet{BoStEnglishHindiTranslation2008},
we found additional out-of-domain language models damaging to BLEU score.

Symmetrical word alignments for grammar extraction were obtained using Giza++
and the scripts accompanying Moses. Alignments were computed on first four
(lowercased) characters of each training token, and the \textit{grow-diag-final-and}
symmetrization heuristic was used.

The results of the hierarchical models trained on various datasets,
compared with classical phrase-based models are shown in \Tref{tab:joshua}.

% - Popsat důležité varianty experimentů, které jsme zkoušeli:
% -- Oficiální testování Ondrovým počítadlem (Ondra se zaručil, že na tokenizaci to nesahá):
%    cp ~bojar/diplomka/granty/hindi/statmt/playground/exp.mert.SRCtides.train+en+lc.TGT+hi+for.c9ce7fbc.-fe.r0-0DEVtides.dev.Sbeam.20090806-1700/testbleu .
%    gunzip -c /net/projects/hindi/data/augmented_corpora/tides.test/hi.gz > test.hi
%    testbleu test.hi < /net/projects/hindi/experiments/joshua/josh11-tmtides-lmtides-lm3-preicon2-prenorm-pretokdz/eval/test.joshua.hi
% -- baseline: trénování na tides (srovnat s Mosesem letos i loni)
%    josh11-tmtides-lmtides-lm3-prebase = 11.28 (Ondra naměřil 11.10±0.79) (pozor, tenhle výsledek nemám s Ondrovým alignmentem lcstem4!)
%    josh11-tmtides-lmtides-lm3-preicon2-prenorm-pretokdz = 12.27 (po normalizaci - kruci, měli bychom ukázat i nějaký výsledek před normalizací, jinak naše tvrzení, že jsme lepší než kdokoli loni, není zajímavé)
%    testbleu: 12.27±0.83
%    Moses: 1*LM 2*tides.train                                                     	11.46±0.72
% -- trénování na tides a daniel pipes
%    ac-tmtides+dpipes-lmtides-lm3 (allcstem4, ale tehdy jsem to ještě do názvu pokusu nepsal)
%    = 12.58
%    testbleu: 12.58±0.77
%    Moses: 1*LM 2*tides.train 1*dani                                              	11.93±0.75
% -- trénování na tides, daniel pipes a emille
%    ac-allcstem4-tmtides+dpipes+omille-lmtides-lm3 = 11.32
%    testbleu: 11.32±0.74
%    Moses: 1*LM 2*tides.train 1*dani 1*emille-om                                  	10.06±0.72
% -- trénování na tides, daniel pipes a slovník (zkusit ještě ten opravený?!!!)
%    ac-allcstem4-tmtides+dpipes+dictfilt-lmtides-lm3 = 12.43 (to je starý neopravený a malý slovník)
%    testbleu: 12.43±0.79
%    Moses: 1*LM 2*tides.train 1*dani 1*dictfilt                                   	11.90±0.78
% -- ke všem číselným výsledkům v této kapitole by to chtělo hned vedle vidět odpovídající výsledky Mosese.
% PS: V sekci o Joshuovi by se mi libilo, kdybys to nejak propojil s lonskym reorderingek: ze vlastne podstata SCFG je svym zpusobem syntakticky transfer, tedy to, co jsme vloni hrube resili reorderingovymi pravidly.

\begin{table}[ht]
\begin{centering}
\begin{tabular}{l|l|l}
\textbf{Parallel data} & \textbf{Joshua} & \textbf{Moses} \\
\hline
Tides & 12.27±0.83 & 11.46±0.72\\
Tides+DP & \textbf{12.58±0.77} & 11.93±0.75\\
Tides+DP+Emille & 11.32±0.74 & 10.06±0.72\\
Tides+DP+Dict & 12.43±0.79 & 11.90±0.78\\
\end{tabular}
\caption{Results of Joshua compared with Moses}
\label{tab:joshua}
\end{centering}
\end{table}



\section{Results}
\label{results}

As far as automatic evaluation is concerned, the best result reported on in
this paper is the 12.58 BLEU of Joshua trained on Tides and Daniel Pipes
(\Tref{tab:joshua}). Moses was not able to outperform this score despite
its ability to learn factored models. The best Moses score is 12.22 (morphology).
%(in
%tandem with the Mumbai tagger, see \Tref{tab:morph}) 

The greatest mystery is the fact that adding Emille to the training data
does not improve results with either system.
% Joshua: tides+dpipes+omille: 0.1298 dev, 0.1132 test.
% Joshua/Emille/Gaurav:        0.1285 dev, 0.1045 test.
% Bez omilla (tides+dpipes):   0.1124 dev, 0.1258 test.
Although we are not able to explain this in full, we made a few observations:
%\begin{itemize}
%  \item

{\bf 1.}
When asked to extract its SFCG, Joshua needs to see the source
        (English) side of the data to be decoded, and extracts only the rules
        relevant to the dataset. Thus we extract one grammar for the
        development data (used for minimum error rate training) and another
        for the test data. While the development grammar changes when Emille
        is added, the test grammar remains the same. As if Emille was unable
        to approximate the test data in any way.
%  \item

{\bf 2.}
MERT optimizes 5 feature weights of Joshua: the language model
        probability, the word penalty (preference for shorter translations),
        and three translation features: $P(e|f)$, $P_{lex}(f|e)$ and
        $P_{lex}(e|f)$. When Emille is involved, MERT always pushes the
        non-lexical translation probability extraordinarily high, and causes
        overfitting. While for other experiments we usually saw better BLEU
        scores on test data than on development data, the opposite was the case
        with Emille.
%\end{itemize}

\section{Related Research}


% Tohle zamlcime, stejne jsme v tabulce.
%\perscite{BoStEnglishHindiTranslation2008} tried to improve a statistical
%phrase-based baseline MT
%system by adding more data, standard lexicalized and a heuristical rule-based
%reordering and an explicit modeling of (unsupervised) morphological coherence
%on the target side. Neither of the approaches helped much.

\perscite{pushpak:2009} improve Hindi morphology by transfering additional
information from English parse tree and semantic roles, in addition to some
hard-coded syntax-based reordering. However, they use an unspecified corpus
making their results incomparable.

% XXX reorderingy? no jen jednoho zastupce, a jako ze Joshua leccos resi sam

%\subsection{Previous Work}
\label{previous}

The ICON 2008 NLP Tools Contest \citep{nlptoolscontest:2008:summary}
included translation of Tides test data. Although our retokenized and cleaned
data make a direct comparison impossible, our preliminary experiments on the
unclean data are comparable. In \Tref{tab:previous} we compare four results
presented at ICON 2008 with our hierarchical model trained on (unclean) Tides,
with a Tides trigram language model.

\begin{table}[ht]
\begin{centering}
\begin{tabular}{l|r}
\textbf{System} & \textbf{BLEU} \\
\hline
Mumbai \citep{Damani:2008} & 8.53\\
Kharagpur \citep{goswami:2008} & 9.76\\
Prague \citep{BoStEnglishHindiTranslation2008} & 10.17\\
Dublin \citep{srivastava:2008} & 10.49\\
present Joshua & 11.10\\
\end{tabular}
\caption{Previous work compared to our hierarchical model (Joshua) on unclean data}
\label{tab:previous}
\end{centering}
\end{table}



\section{Human Evaluation}
\label{human}



The richer morphology and a high degree of reordering are known to render BLEU
unreliable. While we still optimize for BLEU, our manual analysis reveals
relatively low correlation. An extensive study similar to
\cite{callisonburch-EtAl:2009:WMT} would be very valuable.

We were able to conduct three small-scale manual evaluations. Our
annotator was given the source English text and four or five Hindi translations in a
randomized order of 100
sentences (each time a different subset of the full test set). He assigned a
mark to each of the four Hindi translations: giving no
mark  (``0'') indicates a completely incomprehensible translation. A single star
(``*'') denotes sentences with severe errors and hard to understand but still
related to the input. Two stars (``**'') are assigned to more or less
acceptable translations, possibly with many errors but understandable and
conveying most of the original meaning.

\Tref{tab:human} evaluates some basic configurations of Moses. For a comparison,
we include the reference translations (REF) among the hypotheses (due to the
randomization, the annotator cannot be entirely sure which of the hypotheses is
the reference but often it can be guessed from the striking difference in
quality). We see that even 6 reference translations were marked as inadequate
and 11 as very bad. Because the test set is a part of the Tides corpus, these
figures give a rough estimate of the overall corpus quality.

The remaining figures in \Tref{tab:human} document that (while somewhat
suspicious by itself), the Tides training data are most informative with respect
to the (Tides) test set: a system trained completely out-of-domain on everything
except Tides was able to deliver only 3 acceptable translations (OOD). The TIDP
and WC10 systems compare the effect of adding more parallel data (TIDP for
Tides+DanielPipes) vs. adding the unsupervised morphological factor (WC10, 10
word classes). We observe that the difference indicated by human evaluation is
rather convincing: nearly twice as many acceptable sentences in TIDP, but
negligible in BLEU. This confirms our doubts about the utility of BLEU scores
for languages with rich morphology and the neccessity to regularly run manual
evaluations.


\begin{table}[ht]
\begin{centering}
\begin{tabular}{l|r|r|r|r}
{\bf System} & {\bf 0} & {\bf*} & {\bf**} & {\bf BLEU}\\
\hline
REF  &  6    &  11  &   83   &  -- \\
OOD &   80  &   17   &  3   &   1.85±0.24\\
TIDP &  26  &   44   &  30   &  11.93±0.75\\
WC10 &  38   &  46   &  16   &  11.76±0.74\\
\end{tabular}
\caption{Manual evaluation of some basic Moses setups: training out of domain
(OOD) and adding either more parallel data (TIDP) or a factor for morphological
coherence (WC10) compared to the quality of the reference translation (REF).
}
\label{tab:human}
\end{centering}
\end{table}

In \Tref{tab:human:mojo}, we evaluate the difference between phrase-based
(Moses) and hierarchical (Joshua) translation model. Both systems are trained in
identical conditions: both the translation and the language model are trained on
Tides and DanielPipes. For Moses, we wanted to confirm the observation that more
data are more important than ensuring morphological coherence on another 100
manually judged sentences. Therefore, we report also Moses-DPipes+POStags, a
setup trained on Tides only but including the supervised morphology of Mumbai
tagger.

We see that while the BLEU scores indicate the superiority of the hierarchical
model over the phrase-based model with lexicalized reordering, the difference in
manual judgments seems less convincing. Only 3
sentences were ranked better. On the other hand, we confirm that even the
supervised morphology is less important than a good parallel data source.

\begin{table}[ht]
\begin{small}
\begin{centering}
\begin{tabular}{l|r|r|r|r}
{\bf System} & {\bf 0} & {\bf*} & {\bf**} & {\bf BLEU}\\
\hline
REF   &  6   &   10    &  84   &  -- \\ 
Joshua  &  32   &  37   &  31   &   12.58±0.77\\
Moses  &  35  &   35   &  30   &  11.93±0.75\\
%Moses+DPipesR  &  33  &   42  &   25  &   11.80±0.73\\
% Moses+DPipesR je reduced English, zatajuju
Moses-DPipes+POStags  & 32  &   42   &  26   &  12.03±0.75\\
\end{tabular}
\caption{%
Manual judgements of hierarchical (Joshua) vs. phrase-based (Moses) translation.
}
\label{tab:human:mojo}
\end{centering}
\end{small}
\end{table}

\Tref{tab:human:more} provides manual analysis of our baseline Moses setup
(simple
phrase-based translation, lexicalized reordering, language model trained on the
full target side of the parallel data) trained on various subsets of
our parallel data. We start with the combination of Tides and DanielPipes
(TIDP), add Emille (EM), all other corpus sources (oth) and finally the
dictionary Shabdanjali in two variants: full (DICTFull) and filtered to contain
only Hindi words confirmed in a big monolingual corpus (DICTFilt). The BLEU
scores indicate that Emille hurts the performance when tested on Tides test
set. This surprising result was indeed confirmed by the manual analysis: only 12
instead of 19 sentences were translated acceptably. Adding further data reduces
the detrimental effect of Emille (not observed in BLEU scores) but the best
performance is achieved by
Tides+DanielPipes only.  Note that this final manual evaluation was based on a
smaller dataset of 53 sentences only.

\begin{table}[ht]
\begin{small}
\begin{centering}
\begin{tabular}{l|r|r|r|r}
{\bf System} & {\bf 0} & {\bf *} & {\bf**} & {\bf BLEU}\\
\hline
REF         &  0 &          8   &   45   & -- \\
TIDP           &  20 &       14   &  19   &11.89±0.76\\
TIDPEM        &   22 &          19   &   12   &9.61±0.75\\
TIDPEMoth      &   17 &         25  &    11   & 10.97±0.79\\
TIDPEMothDICTFilt     &    23 &  17  &    13  & 10.96±0.75\\
TIDPEMothDICTFull    &    22 &   16  &    15  & 10.89±0.69 \\
\end{tabular}
\caption{The effect of additional (out-of-domain) parallel data in phrase-based
translation.}
\label{tab:human:more}
\end{centering}
\end{small}
\end{table}   


\section{Conclusions and Future Work}
\label{sec:concl}
%Do závěru bychom mohli plácnout něco jako ``We believe that most techniques
%described in this paper are straightforwardly generalizable to other Indian SOV
%languages.''

We have tried several ways to improve state-of-the-art in English-Hindi SMT,
however results are mixed: 

There are quite a few strange results: 
\begin{itemize}
\item POS tagging does not give better results than automatic word classes.
Hindi textbook-based manual word classes were even better.
\item Adding more training data to Tides either helps insignificantly, or (more
often) hurts BLEU score. 
\item BLEU may not correlate with human judgement. Our limited experiments show
that adding training data may hurt BLEU but improve quality by human judgement.
%\item It has been shown before by \citet{BoStEnglishHindiTranslation2008} that 
%big monolingual Hindi data for LM didn't help either.
\end{itemize}

In our future work we want to further explore problems with existing datasets,
the use of morphology, and the relation of output quality measured in terms of
BLEU vs.
human
judgement. We also believe that there is room for improvement in the quality and
amount of available parallel data.

On the other hand, we have shown that our hand-crafted word classes and some
additional data help Moses achieve significantly better results than reported
previously. Hierarchical decoder Joshua can capture word order even better than
Moses. Its results are always slightly better. And as far as we know, our
current results are the best that have been reported on this dataset.

%\XXX -- někde zmínit, že "čištění" (ve smyslu Mosese, tedy dlouhých vět)
%výsledky kazí, protože nám zmenšuje trénovací data. O to víc je vlastně
%zvláštní, že přidání trénovacích dat typu Emille výsledky taky kazí.


\begin{small}
\bibliography{paper}
\end{small}

\end{document}
